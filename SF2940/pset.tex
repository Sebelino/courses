\documentclass{article}
\title{Problem set, SF2940}
\author{Ville Sebastian Olsson}
\usepackage[a4paper,margin=2em]{geometry}
\usepackage{amsmath}
\usepackage{amsfonts}
\usepackage{amssymb}
\usepackage{amsthm}
\usepackage[parfill]{parskip}
\usepackage{sebelino-mathlib} % Custom sty file
\setcounter{secnumdepth}{0}
\begin{document}
\maketitle

\section{Quiz 1.2}

Let \((\Omega, \mathcal{A}, P)\) be a probability space.
Let \(A\in \mathcal{A}\) and \(B\in \mathcal{A}\)
with \(A\subseteq B\).

\textbf{a)} Is this statement correct?:
\[P(B^\complement \cap A) = P(B)-P(A)\]

\textbf{Solution:}

No. Let \(A=\varnothing\) and \(P(B)>0\).

\begin{align*}
	 & LHS \\
	=& P(B^\complement \cap A) \\
	=& P(B^\complement \cap \varnothing) \\
	=& P(\varnothing) \\
	=& 0 \\
\end{align*}
while
\begin{align*}
	  & RHS \\
	 =& P(B)-P(A) \\
	 =& P(B)-P(\varnothing) \\
	 =& P(B)-0 \\
	 =& P(B) \\
	 >& 0
\end{align*}

\textbf{b)} Is this statement correct?:
\[P(B \cap A^\complement) = P(B)-P(A)\]

\textbf{Solution:}

Yes.
\begin{proof}
\begin{align*}
	 & P(B\cap A^\complement) \\
	=& P(B\setminus A) \\
	=& P(B)-P(A) & (\text{ since }A\subseteq B) \\
\end{align*}
\end{proof}

\section{Quiz 1.5}

Assume
\(P(A) \geq 1-\delta\) and \(P(B) \geq 1-\delta\) for some small \(\delta\geq0\).

Show that \(P(A\cap B) \geq 1-2\delta\).

\textbf{Solution:}

\begin{align*}
	 & P(A\cup B) \\
	=& P(A)+P(B)-P(A\cap B) \\
	\Rightarrow& \\
	 & P(A\cap B) \\
	=& P(A)+P(B)-P(A\cup B) \\
	\geq& (1-\delta)+P(B)-P(A\cup B) & (\text{since }P(A) \geq 1-\delta) \\
	\geq& (1-\delta)+(1-\delta)-P(A\cup B) & (\text{since }P(B) \geq 1-\delta) \\
	\geq& (1-\delta)+(1-\delta)-1 & (\text{since }P(A\cup B) \leq 1) \\
	=& 1-2\delta
\end{align*}

\section{Gut, 3.1, p.24}

Show that if \(X \in C(0,1)\), then so is \(1/X\).

\textbf{Solution:}

\(X\) and \(1/X\) have the same distribution if \(f_X=f_{1/X}\):

\[f_X(x;x_0,\gamma) = \frac{1}{\pi\gamma(1+\frac{(x-x_0)^2}{\gamma^2})}\]
\[f_X(x;0,1) = \frac{1}{\pi \cdot 1 \cdot (1+\frac{(x-0)^2}{1^2})}\]
\[f_X(x;0,1) = \frac{1}{\pi (x^2+1)}\]
Let \(Y=g(X)\) where \(g: \mathbb{R}\setminus\{0\}\to\mathbb{R}\setminus\{0\}\), \(g(x) = \frac{1}{x}\).

\(g\) is bijective, strictly monotone, and its inverse is:
\[g^{-1}(x) = \frac{1}{x}\]

\begin{align*}
 & f_{1/X}(y;0,1) \\
=& f_Y(y;0,1) \\
=& f_X(g^{-1}(y);0,1) \cdot |\frac{d}{dy}g^{-1}(y)| \\
=& \frac{1}{\pi (g^{-1}(y)^2+1)} \cdot |\frac{d}{dy}\frac{1}{y}| \\
=& \frac{1}{\pi ((\frac{1}{y})^2+1)} \cdot |-\frac{1}{y^2}| \\
=& \frac{1}{\pi (\frac{1}{y^2}+1)} \cdot \frac{1}{y^2} \\
=& \frac{1}{\pi (1+y^2)} \\
=& f_X(y;0,1) \\
\end{align*}

\(\therefore f_{1/X}\in C(0,1)\)

\section{Gut, Exercise 1.1, p.57}

Let \(X_1,X_2 \sim U(0,1)\), iid.

Find the distribution of \(X_1+X_2\).

\textbf{Solution:}

Let \(Y=X_1+X_2\).

\begin{align*}
	 & f_Y(y) \\
	=& \int_{-\infty}^\infty f_{X_1,X_2}(x_1,y-x_1)dx_1 \\
	=& \int_{-\infty}^\infty f_{X_1}(x_1)f_{X_2}(y-x_1)dx_1 & (\text{ since } X_1,X_2\text{ iid.}) \\
	=& \int_{-\infty}^\infty \casesii{1}{0\leq x_1 \leq 1}{0}f_{X_2}(y-x_1)dx_1 \\
	=& \int_0^1 \casesii{1}{0\leq x_1 \leq 1}{0}f_{X_2}(y-x_1)dx_1 \\
	=& \int_0^1 1\cdot f_{X_2}(y-x_1)dx_1 \\
	=& \int_0^1 f_{X_2}(y-x_1)dx_1 \\
	=& \int_0^1 \casesii{1}{0\leq y-x_1\leq 1}{0}dx_1 \\
	=& \int_0^1 \casesii{1}{y-1\leq x_1\leq y}{0}dx_1 \\
	=& \int_0^1 \casesii{1}{y-1\leq x_1\leq y\wedge 0\leq x_1 \leq 1}{0}dx_1 \\
	=& \int_0^1 \casesii{1}{\max(0,y-1)\leq x_1\leq \min(1,y)}{0}dx_1 \\
	=& \casesii{\int_0^1 \casesii{1}{\max(0,y-1)\leq x_1\leq \min(1,y)}{0}dx_1}{y \leq 1}{\int_0^1 \casesii{1}{\max(0,y-1)\leq x_1\leq \min(1,y)}{0}dx_1} \\
	=& \casesii{\int_0^1 \casesii{1}{0\leq x_1\leq y}{0}dx_1}{y \leq 1}{\int_0^1 \casesii{1}{y-1\leq x_1\leq 1}{0}dx_1} \\
	=& \casesii{\int_0^y \casesii{1}{0\leq x_1\leq y}{0}dx_1}{y \leq 1}{\int_{y-1}^1 \casesii{1}{y-1\leq x_1\leq 1}{0}dx_1} \\
	=& \casesiii{\int_0^y 1dx_1}{0\leq x_1 \leq y \wedge y\leq 1}{\int_{y-1}^1 1dx_1}{y-1\leq x_1 \leq 1 \wedge \neg(y\leq 1)}{0} \\
	=& \casesiii{\int_0^y 1dx_1}{0\leq y \wedge y\leq 1}{\int_{y-1}^1 1dx_1}{y-1\leq 1 \wedge y>1}{0} \\
	=& \casesiii{\int_0^y 1dx_1}{0\leq y\leq 1}{\int_{y-1}^1 1dx_1}{1<y\leq 2}{0} \\
	=& \casesiii{\hakparen{x_1}_0^y}{0\leq y\leq 1}{\hakparen{x_1}_{y-1}^1}{1<y\leq 2}{0} \\
	=& \casesiii{y-0}{0\leq y\leq 1}{1-(y-1)}{1<y\leq 2}{0} \\
	=& \casesiii{y}{0\leq y\leq 1}{2-y}{1<y\leq 2}{0} \\
\end{align*}

\section{Modelltenta Problem 1}

Let \(f : \mathbb{R} \to \mathbb{R}\) be a Borel function, and \(X\) be a
random variable. Let \(Y = f(X)\).

Prove that \(Y\) is a random variable.

\textbf{Solution:}

\begin{proof}
Let \(X:\Omega\to \mathbb{R}\).
Let \(\mathcal{A}\) be a \(\sigma\)-algebra on \(\Omega\).
Let \(\mathcal{B}\) be the Borel algebra on \(\mathbb{R}\).

\(Y\) is a random variable if \(Y\) is a measurable function w.r.t. measurable
spaces \((\Omega, \mathcal{A})\), \((\mathbb{R}, \mathcal{B})\).

\(Y\) is such a measurable function if \(\forall B\in \mathcal{B}: Y^{-1}(B)\in \mathcal{A}\).

Let
\[B\in \mathcal{B}\tag{I}\]

We need to prove that \(Y^{-1}(B)\in \mathcal{A}\).

\begin{align*}
	 & Y^{-1}(B) \\
	=& \{\omega\in \Omega: Y(\omega)\in B\} \\
	=& \{\omega\in \Omega: f(X(\omega))\in B\} \\
	=& \{\omega\in \Omega: f^{-1}(f(X(\omega)))\in f^{-1}(B)\} \\
	=& \{\omega\in \Omega: X(\omega)\in f^{-1}(B)\} \\
	=& X(f^{-1}(B)) \tag{II} \\
\end{align*}

Since \(X\) is a random variable:
\[\forall B\in \mathcal{B}: X^{-1}(B)\in \mathcal{A} \tag{III}\]

The preimage of \(f:\mathbb{R}\to\mathbb{R}\) is \(f^{-1}: \mathcal{B} \to\mathcal{B}\), so:
\[\forall B \in \mathcal{B}: f^{-1}(B)\in \mathcal{B} \tag{IV}\]

(I) and (IV) imply:
\[f^{-1}(B) \in \mathcal{B} \tag{V}\]

(V) and (III) imply:
\[X^{-1}(f^{-1}(B))\in \mathcal{A}\tag{VI}\]

(VI) and (II) imply:
\[Y^{-1}(B) \in \mathcal{A}\]

So:
\[\forall B\in \mathcal{B}:Y^{-1}(B) \in \mathcal{A}\]

Therefore, \(Y\) is a measurable function.

Therefore, \(Y\) is a random variable.
\end{proof}


\end{document}
