\documentclass{article}
\title{Problem set, SF2940}
\author{Ville Sebastian Olsson}
\usepackage[a4paper,margin=2em]{geometry}
\usepackage{amsmath}
\usepackage{amsfonts}
\usepackage{amssymb}
\usepackage[parfill]{parskip}
\setcounter{secnumdepth}{0}
\begin{document}
\maketitle

\section{Quiz 1.2}

Let \((\Omega, \mathcal{F}, P)\) be a probability space.
Let \(A\in \mathcal{F}\) and \(B\in \mathcal{F}\)
with \(A\subseteq B\).

\textbf{a)} Is this statement correct?:
\[P(B^\complement \cap A) = P(B)-P(A)\]

\textbf{Solution:}

No. Let \(A=\varnothing\) and \(P(B)>0\).

\begin{align*}
	 & LHS \\
	 & P(B^\complement \cap A) \\
	=& P(B^\complement \cap \varnothing) \\
	=& P(\varnothing) \\
	=& 0 \\
\end{align*}
while
\begin{align*}
	  & RHS \\
	  & P(B)-P(A) \\
	 =& P(B)-P(\varnothing) \\
	 =& P(B)-0 \\
	 =& P(B) \\
	 >& 0
\end{align*}

\section{Quiz 1.5}

Assume
\(P(A) \geq 1-\delta\) and \(P(B) \geq 1-\delta\) for some small \(\delta\geq0\).

Show that \(P(A\cap B) \geq 1-2\delta\).

\textbf{Solution:}

\begin{align*}
	 & P(A\cup B) \\
	=& P(A)+P(B)-P(A\cap B) \\
	\Rightarrow& \\
	 & P(A\cap B) \\
	=& P(A)+P(B)-P(A\cup B) \\
	\geq& (1-\delta)+P(B)-P(A\cup B) & (\text{since }P(A) \geq 1-\delta) \\
	\geq& (1-\delta)+(1-\delta)-P(A\cup B) & (\text{since }P(B) \geq 1-\delta) \\
	\geq& (1-\delta)+(1-\delta)-1 & (\text{since }P(A\cup B) \leq 1) \\
	=& 1-2\delta
\end{align*}

\section{3.1}

Show that if \(X \in C(0,1)\), then so is \(1/X\).

\textbf{Solution:}

\(X\) and \(1/X\) have the same distribution if \(f_X=f_{1/X}\):

\[f_X(x;x_0,\gamma) = \frac{1}{\pi\gamma(1+\frac{(x-x_0)^2}{\gamma^2})}\]
\[f_X(x;0,1) = \frac{1}{\pi \cdot 1 \cdot (1+\frac{(x-0)^2}{1^2})}\]
\[f_X(x;0,1) = \frac{1}{\pi (x^2+1)}\]
Let \(Y=g(X)\) where \(g: \mathbb{R}\setminus\{0\}\to\mathbb{R}\setminus\{0\}\), \(g(x) = \frac{1}{x}\).

\(g\) is bijective, strictly monotone, and its inverse is:
\[g^{-1}(x) = \frac{1}{x}\]

\begin{align*}
 & f_{1/X}(y;0,1) \\
=& f_Y(y;0,1) \\
=& f_X(g^{-1}(y);0,1) \cdot |\frac{d}{dy}g^{-1}(y)| \\
=& \frac{1}{\pi (g^{-1}(y)^2+1)} \cdot |\frac{d}{dy}\frac{1}{y}| \\
=& \frac{1}{\pi ((\frac{1}{y})^2+1)} \cdot |-\frac{1}{y^2}| \\
=& \frac{1}{\pi (\frac{1}{y^2}+1)} \cdot \frac{1}{y^2} \\
=& \frac{1}{\pi (1+y^2)} \\
=& f_X(y;0,1) \\
\end{align*}

\(\therefore f_{1/X}\in C(0,1)\)

\end{document}
